\begin{section}{Problem Overview}
 Integrated healthcare scheduling involves the coordination of resources
 across multiple services within a unified healthcare system. Its primary
 goal is to streamline and optimize patient flow throughout various departments
 and facilities within the hospital. The advantages of integrated healthcare
 optimization are numerous, including an enhanced patient experience, increased
 operational efficiency, and better utilization of resources across the entire
 hospital network.

 The \textit{Integrated Healthcare Timetabling Problem} (IHTP) \cite{ihtc2024} transpose the
 aforementioned concept into a combinatorial optimization problem, bringing together
 three distinct operational challenges:
 Patient-to-Room Assignment (PRA), Nurse-to-Patient Assignment (NPA) and
 Surgical Case Planning (SCP).
 More precisely, the IHTP requires to establish \textit{(i)} the date of admission for patients,
 \textit{(ii)} the room where they are admitted and \textit{(iii)} operating theater where they are operated,
 \textit{(iii)} the room nurses are assigned to during each shift.
 The problem is subject to several constraints, specific to a sub-problem or arising from their interaction: they divide
 into hard and soft constraints, the former being mandatory and
 the latter being subject to penalties when violated. For practical purposes, we will refer to problem solutions
 where all hard constraints are satisfied as a \textit{stable}, \textit{unstable} otherwise.

 Given a test instance, all the information required are provided,
 including pre-admitted occupants, patients, nurses, surgeons,
 operating theaters, rooms, shifts, scheduling dimension in days and penalty weights for soft constraints.


\end{section}