\begin{section}{Problem Overview}
 Integrated healthcare scheduling involves the coordination of resources
 across multiple services within a unified healthcare system. Its primary
 goal is to streamline and optimize patient flow throughout various departments
 and facilities within the hospital. The advantages of integrated healthcare
 optimization are numerous, including an enhanced patient experience, increased
 operational efficiency, and better utilization of resources across the entire
 hospital network.

 The \textit{Integrated Healthcare Timetabling Problem} (IHTP) \cite{ihtc2024} transpose the
 aforementioned concept into a combinatorial optimization problem, bringing together
 three distinct operational challenges:
 Patient-to-Room Assignment (PRA), Nurse-to-Patient Assignment (NPA) and
 Surgical Case Planning (SCP).
 More precisely, the IHTP requires to establish \textit{(i)} the date of admission for patients,
 \textit{(ii)} the room where they are admitted and \textit{(iii)} operating theater where they are operated,
 \textit{(iii)} the room nurses are assigned to during each shift.
 The problem is subject to several constraints, specific to a sub-problem or arising from their interaction.
 The constraints are the following:
 \begin{itemize}
     \item \textit{Hard Constraints}, which must be satisfied at all costs:
           \begin{itemize}
               \item No gender mixing in rooms;
               \item Patients can be assigned only to a compatible room;
               \item Surgeon cannot work overtime;
               \item Operating theaters capacity cannot be exceeded;
               \item Mandatory patients need to be assigned during the provided period;
               \item Patients cannot be scheduled before their release date;
               \item Room capacity cannot be exceeded;
               \item Rooms with patients assigned must be covered.
           \end{itemize}
     \item \textit{Soft Constraints}, which can be violated at the cost of a penalty:
           \begin{itemize}
               \item Age groups mixing in rooms;
               \item Nurses require a minimum skill level to cover a certain patient.
               \item Continuity of care for patients from nurses;
               \item Nurses have a maximum workload during a shift;
               \item Number of open operating theaters per day should be minimized;
               \item Number of surgeon transfer between operating theaters should be minimized;
               \item Patients should be admitted the earliest possible;
               \item Number of unscheduled non-mandatory patients should be minimized.
           \end{itemize}
 \end{itemize}

 Given a test instance, all the information required are provided,
 including pre-admitted occupants, patients, nurses, surgeons,
 operating theaters, rooms, shifts, scheduling dimension in days and penalty weights for soft constraints.


\end{section}