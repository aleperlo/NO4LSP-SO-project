\begin{section}
 {Discussion}
 The proposed approach performances are overall satisfactory, with the algorithm being able to find a solution
 guaranteeing no hard constraint violation in all the tested instances. The penalty values reached by TS
 are sometimes competitive with the precomputed solutions, but are generally higher, especially for the
 larger problems.

 Focusing on the first instances, we notice that they achieve great results in term of patient scheduling,
 with an high percentage of penalty due to
 non minimization of nurses soft constraints like continuity of care. As mentioned before, this happens
 when the tabu list size is too large: however, in this cases, tests performed with smaller tabu list size
 did not lead to better results, but to a higher penalty value or even violation of hard constraints.
 One possible solution to this issue could be to implement the possibility for nurse substitution,
 without the need to free up the room and remove the assigned nurse beforehand: this would allow the algorithm
 to maintain stability and optimize nurse assignment in a more effective way. It is worth mentioning, however,
 that this would require a more complex neighborhood definition and a more sophisticated tabu list management,
 since the algorithm would need to consider the possibility of swapping nurses between rooms and shifts.
 As a matter of fact, a bad management of this feature could lead to the algorithm cycling between nurses interchange,
 significantly worsening the performances.
 For this reasons, an implementation of this kind would surely translate in a more complex and slower algorithm.

\end{section}