\begin{section}
 {Approach}
 The proposed approach involves the application of the \textit{Tabu Search} (TS) algorithm, which is
 it is a local search method that aims to find the global optimum of a combinatorial optimization problem. \cite{ts}


 The main idea behind TS is to enhance local search performances by relaxing its rules: in particular,
 worsening moves can be accepted if no improving move is available, which prevent from getting stuck in local optima,
 and previously-visited solutions are discouraged, which prevent cycles.
 This last aspect is
 implemented storing moves in a queue, called \textit{tabu list}, the dimension of which needs to be tuned.
 In order to mitigate the constraints introduced bu the list, the algorithm also includes the possibility
 to accept forbidden moves when they met a certain \textit{aspiration criterion}, for example if they lead to
 a better solution than the current one.

 The algorithm is the following:
\begin{lstlisting}[language=c]
    algorithm TabuSearch(S, maxIter, f, neighborhoods, aspirationCriteria):
    // INPUT
    //    S = the search space
    //    maxIter = the maximal number of iterations
    //    f = the objective function
    //    neighborhoods = the definition of neighborhoods
    //    aspirationCriteria = the aspiration criteria
    // OUTPUT
    //    The best solution found

    Choose the initial candidate solution s in S
    s* <- s  // Initialize the best-so-far solution
    k <- 1

    while k <= maxIter:
        // Sample the allowed neighbors of s
        Generate a sample V(s, k) of the allowed solutions in N(s, k)
        // s' in V(s, k) iff (s' not in T) or (a(k, s') = true)

        Set s to the best solution in V(s, k)

        // Update the best-so-far if necessary
        if f(s) < f(s*):
            s* <- s

        Update T

        // Start another iteration
        k <- k + 1

    return s*
\end{lstlisting}


 The next paragraphs will discuss the main aspects that need to be addressed in order to adapt TS to IHTP.


 \begin{subsection}
     {Solution Representation}
     The current status of the problem is represented by
     three main data structures:
     \begin{itemize}
         \item \textit{Patient Admission and Scheduling (PAS)}:
               boolean array of size $n_{\text{days}} \times n_{\text{rooms}} \times n_{\text{patients}}$.
               For each patient, it indicates which room they are assigned to during all the days of recovery.
         \item \textit{Surgical Case Planning (SCP)}:
               boolean array of size $n_{\text{days}} \times n_{\text{patients}} \times n_{\text{surgeons}} \times n_{\text{operating theaters}}$.
               For each patient, it indicates the day of admission (i.e. surgery) along with the predetermined surgeon and the chosen operating theater.
         \item \textit{Nurse-to-Room Assignment (NRA)}:
               boolean array of size $n_{\text{shifts}} \times n_{\text{rooms}} \times n_{\text{nurses}}$.
               For each nurse, it indicates which room they are assigned to during each shift.
     \end{itemize}
     In order to lower computational time of certain soft constraints, two other structures are used:
     \textit{Workload} and \textit{Skill level}, both arrays of size $n_{\text{shifts}} \times n_{\text{rooms}} \times n_{\text{patients}}$,
     storing the workload and skill level required by a patient assigned to a room during a certain shift.
 \end{subsection}

 \begin{subsection}
     {Starting Point}
     As already mentioned, a test instance start with some patients already admitted to the hospital, called occupants.
     This configuration clearly is an unstable status of the problem, since no hard constraint is guaranteed to be satisfied.
     With these premises, two approaches can be considered in order to generate a starting point: create a stable configuration from scratch
     using advanced and exhaustive combinatorial techniques, which is a very complex task both
     in terms of time and space complexity and would partly defy the purpose of this work;
     otherwise, accept the unstable configuration as it is and let TS
     find a stable solution by opportunely tuning its behavior.
     The second approach is the one that has been chosen, since it is more efficient and allows to fully grasp
     the potential of the algorithm. The steps taken to direct the algorithm in a direction of stability are presented
     in the next paragraphs.

 \end{subsection}

 \begin{subsection}
     {Neighborhood Definition}
     Given the nature of the problem and the constraints involved,
     the type of moves performed by the algorithm are the following:
     \begin{itemize}
         \item Patient
               \begin{itemize}
                   \item Scheduling: for a patient, admit them to a room on a day and establish the operating theater.
                   \item Unscheduling: for a patient, remove them complexly from the schedule.
               \end{itemize}
         \item Nurse
               \begin{itemize}
                   \item  Assignment: for a nurse, assign them to a room during a shift.
                   \item Unassignment: for a nurse, remove them from an assigned room during a shift.
               \end{itemize}
     \end{itemize}
     The neighborhood is defined as the set of all possible moves that can be performed
     on the current solution, meaning that for each of the aforementioned types of moves
     all the possible combinations of patients, rooms, days, operating theaters, nurses and shifts are considered.
     Obviously, moves effectively considered are those that do not violate hard constraints and do not destabilize the solution.
     Moreover, given that the current status of the problem is not guaranteed to be stable, the neighborhood
     needs to be ulteriorly filtered if necessary: more precisely, when there are still non-admitted mandatory patients,
     these have priority over the others and scheduling regarding optional patients is ignored.


     Since TS operates comparing the current moves with the ones stored in the tabu list, it is also necessary to
     define a way to compare two moves. For patient moves, the comparison is performed by checking
     the type and all the parameters involved (i.e. patient, room, day, operating theater),
     meaning that only the same exact action is considered tabu.
     For nurse moves, on the other hand, equality is asymmetric and comparison follows these rules:
     when considering an assignment, it can only be equal to an assignment with the same parameters (i.e. nurse, room, shift),
     so an exact copy;
     when considering an unassignment, it can be equal to both an assignment or unassignment with equal nurse and room,
     independently of the shift.
     The reason behind this choice is that the nurses assignment is a vital part of the schedule,
     since room coverage needs to be granted in order to admit patients:
     the strategy is to force the algorithm in a direction where nurses are more likely to be assigned and avoids the situation
     where nurses are assigned and unassigned multiple times consecutively just because the shift changes.

     As a consequence of all these choices, we expect the optimization to be more stable for hard constraint satisfaction, specifically mandatory patients admission
     and room coverage, but less effective in minimizing soft constraints focusing on nurses.


 \end{subsection}

 \begin{subsection}
     {Tabu List Size}
     The tabu list size is a crucial parameter of the algorithm, depending
     both on the problem dimension and the neighborhood definition we established.
     A small tabu list size can lead to cycling and getting stuck to a local space, while a large one can
     slow down the algorithm and make it less effective.

     In our case, TS evaluates moves and performances based on a
     the penalty of the soft constraints, moreover it has been adapted to consider
     the hard constraints as well. As a consequence,
     the two previous edge cases translates as follows:
     \begin{itemize}
         \item Small tabu list: the algorithm tends to reaccept previous moves and
               performs a small set of actions, more precisely nurses assignment and unassignment.
               In other words, TS focus more on penalty optimization and get stuck in a local space where
               the best action at each step is to assign or unassign a nurse, which slightly improves the penalty
               or keeps it constant. This is not desirable, since the hard constraints are not guaranteed to be satisfied.
         \item Large tabu list: the algorithm tends to avoid previous moves and explores a larger space,
               performing patient scheduling. However, patient unschedule
               actions remain tabu for a long time preventing TS to free up a room, which means
               the current nurse assigned to it is stuck and cannot be exchanged with a
               better nurse in terms of soft constraints.
               This is not desirable as well, since the penalty is not minimized.

     \end{itemize}
     These cases are the extremes of the spectrum,
     but evidence well how the tabu list size is the key parameter in order to
     opportunely balance the algorithm and avoid having
     hard constraints violated, nor them prevailing over the soft ones.


 \end{subsection}

 \begin{subsection}
     {Aspiration Criterion}
     In this case, the aspiration criterion is used to accept forbidden moves when they lead to
     a better solution than the best overall one found so far.


 \end{subsection}
\end{section}